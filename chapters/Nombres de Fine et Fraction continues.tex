\chapter{Définitions et préliminaires}
% Dans ce chapitre, nous allons trouver des relations entre $C_{n}$ et $F_{n}$ en utilisant le
% concept des fractions continues.
\section{Chemins de Motzkin}

\begin{definition}
	\begin{rm}
		Un chemin est une suite de points $(A_{i})_{0 \leq i \leq n}$ dans $\mathbb{N}\times \mathbb{N}$ tel que, pour tout $i$, si $
			(a_{i}, b_{i})$ sont les coordonnées de $A_{i}$, alors pour tout $i<n$, $a_{i+1} - a_{i}=1$ et $|b_{i+1} - b_{i}| \leq 1$.
	\end{rm}
\end{definition}
Dans la suite, on notera par $\Gamma_{n,1}^{i}$ l'ensemble des chemins de $A_{0} = (0,0)$ vers $A_{n} = (n,i)$ en $n$ pas.

\begin{property}
	Soit $c \in \Gamma_{n,1}^{i}$ avec $c=(A_{0},A_{1}, \cdots, A_{n})$ où $A_{i}=(a_{i}, b_{i})$.\\
	On a $b_{0}=0$, $b_{n}=i$ et pour tout $k \in \{0, 1, \cdots, n\}$, $A_{k} = (k, b_{k})$.
\end{property}

\begin{definition} \label{path_char}
	\begin{rm}
		Soit $c \in \Gamma_{n,1}^{i}$. On dit que le $k$-ième pas de $c$ est:
		\begin{itemize}
			\item[$\bullet$] un palier, que l'on note par la lettre $p$, si $b_{k}- b_{k-1} = 0$.
			\item[$\bullet$] une montée, que l'on note par la lettre $m$, si $b_{k}- b_{k-1} = 1$.
			\item[$\bullet$] une descente, que l'on note par la lettre $d$, si $b_{k}- b_{k-1} = -1$.
		\end{itemize}
	\end{rm}
\end{definition}
Tout chemin $c \in \Gamma_{n,1}^{i}$ est entièrement détérminé par $b_{0}, b_{1}, \cdots, b_{n}$.
En conséquence, nous adoptons la notation suivante pour un chemin donné:
$c = c_{1}c_{2}\cdots c_{n}$, où $c_{k} \in \{p, m, d\}$ représente la $k$-ième pas de $c$ de niveau $b_{k-1}$. La longueur du chemin, notée $|c|$, correspond au nombre total de pas. On désigne par $|x|_{u}$ le nombre de lettre dans le mot $x$ égal à $u$. Nous notons également l'ensemble $\{1, 2, \ldots, n\}$ par $[n]$ tout au long du document.

\begin{definition}
	\begin{rm}
		Les chemins de Motzkin sont les éléments de l'ensemble $\Gamma_{n,1}^{0}$ qui vérifient, pour tout $c \in \Gamma_{n,1}^{0}$:
		\begin{itemize}
			\item[$(i)$] $\forall i \in [n]$, $|c_{1}c_{2}\cdots c_{i}|_{m} \geq |c_{1}c_{2}\cdots c_{i}|_{d} $
			\item[$(ii)$] $|c_{1}c_{2}\cdots c_{n}|_{m} = |c_{1}c_{2}\cdots c_{n}|_{d} $
		\end{itemize}
	\end{rm}
\end{definition}
On note par $\gamma_{k-1}$ le niveau du $k$-ième pas d'un chemin de Motzkin $c$. On a $\gamma_{0} = 0$.
\newpage
\begin{proposition} \label{levelOfPath}
	Soit $c \in \Gamma_{n,1}^{0}$. Pour tout $i \geq 2$, on a $|c_{1} \cdots c_{i-1}|_{m} - |c_{1} \cdots c_{i-1}|_{d} = \gamma_{i-1}$
\end{proposition}
Preuve: On a \\
$$
	\begin{array}{c c l}
		\gamma_{i-1} & = & (\gamma_{i-1}-\gamma_{i-2})+(\gamma_{i-2}-\gamma_{i-3})+\cdots+(\gamma_{1}-\gamma_{0})                           \\
		             & = & \underset{j=1}{\overset{i-1}{\sum}}(\gamma_{j}-\gamma_{j-1})                                                     \\
		             & = & \underset{\underset{c_{j}=m}{j\leq i-1}}{\sum}(\gamma_{j}-\gamma_{j-1})+\underset{\underset{c_{j}=d}{j\leq i-1}}
		{\sum}(\gamma_{j}-\gamma_{j-1})+\underset{\underset{c_{j}=p}{j\leq i-1}}{\sum}(\gamma_{j}-\gamma_{j-1})                             \\
		             & = & |c_{1}\cdots c_{i-1}|_m-|c_{1}\cdots c_{i-1}|_d \hspace{10pt}\blacksquare
	\end{array}
$$
\vspace{5pt}
\begin{definition}
	\begin{rm}
		Un 2-chemin de Motzkin est un chemin de Motzkin caractérisé par deux types de paliers : le palier rouge, noté $r$, et le palier bleu, noté $b$. De plus, un chemin $c = c_{1}c_{2}\cdots c_{n}$, avec $c_{k} \in \{m, d, r, b\}$, est considéré comme un 2-chemin de Motzkin s'il ne comporte aucun palier rouge de niveau zéro.
	\end{rm}
\end{definition}

Dans la suite, l'ensemble des 2-chemin de Motzkin en $n$ pas sera noté par $\Gamma_{n}$.
\vspace{5pt}
\begin{definition}
	\begin{rm}
		Un chemin de Fine est un 2-chemin de Motzkin sans palier bleu de niveau zéro. On note par $\mathcal{F}_{n}$ l'ensemble des chemin de Fine en $n$ pas.
	\end{rm}
\end{definition}

\section{Nombres de Catalan}
\begin{definition}
	\begin{rm}
		Les nombres de Catalan sont les nombres $C_{n}$ qui vérifient la relation de récurrence suivante:
		\begin{itemize}
			\item [($i$)] $C_{0}=1$
			\item [($ii$)] $\forall n \geq 1$, $C_{n} = \underset{i=0}{\overset{n-1}{\sum}}C_{i}C_{n-i-1}$
		\end{itemize}
	\end{rm}
\end{definition}

On note $C(x)=\underset{n\geq 0}{\sum}C_{n}x^{n}$ la fonction génératrice ordinaire des nombres de Catalan
\begin{proposition}
	\begin{rm}
		On a \[C(x)=\cfrac{1-
				\sqrt{1-4x}}{2x}\]\\
	\end{rm}
\end{proposition}
Preuve: On a\\\[
	\begin{array}{r r l}
		C(x) & =       & 1+\underset{n\geq1}{\sum}C_{n}x^n = 1+ \underset{n\geq1}{\sum}x^{n}\underset{i=0}{\overset{n-1}{\sum}}C_{i}C_{n-i-1} = 1+
		\underset{n\geq1}{\sum}x^n\underset{\underset{j+i=n-1}{0\leq i, j \leq n-1}}{\sum}C_{i}C_{j}                                               \\
		     & \text{} &                                                                                                                           \\
		     & =       & 1+x\underset{n\geq0}{\sum}\text{ }\underset{\underset{j+i=n}{0\leq i, j \leq n}}{\sum}(C_{i}x^i)(C_{j}x^j)
		= 1 +x\left[\left(\sum_{i\geq0}C_{i}x^i\right)\left(\sum_{j\geq0}C_{j}x^j\right)\right]                                                    \\
		     & \text{} &                                                                                                                           \\
		     & =       & 1+x(C(x))^2
	\end{array}
\]\\
Cela implique que $C(x)$ est solution de l'équation $xt^2-t+1 = 0$. On a $C(x)=\cfrac{1-\sqrt{1-4x}}{2x}$\\ ou $C(x)=\cfrac{1+\sqrt{1-4x}}{2x}$.\\
La limite quand $x$ tends vers $0$ de $\cfrac{1+\sqrt{1-4x}}{2x}$ (resp. $\cfrac{1-\sqrt{1-4x}}{2x}$) est infini (resp. 1) \\
En conséquent, $C(x)=\cfrac{1-\sqrt{1-4x}}{2x}$ \hspace{5pt}$\blacksquare$

\section{Fractions continues de Stieltjes}
\begin{definition}
	\begin{rm}
		Une $S$-fraction est une expression de la forme
		\[
			S(z)=\cfrac{1}{1-\cfrac{c_{1}z}{1-\cfrac{c_{2}z}{1-\cfrac{c_{3}z}{\ddots}}}}
		\]
		où $z$ est une variable formelle et $c_{i}$ sont des éléments d'un anneau commutatif.
	\end{rm}
\end{definition}
La proposition suivante est analogue au Lemme 2.11 dans \cite{ref30}.
\begin{proposition}\label{j-frac}
	\begin{rm}
		\[
			\begin{array}{l l l}

				S(z) & =       & \cfrac{1}{1-c_{1}z-\cfrac{c_{1}c_{2}z^2}{1-(c_{2}+c_{3})z-\cfrac{c_{3}c_{4}z^2}{1-(c_{4}+c_{5})z-\cfrac{c_{5}c_{6}z^2}{\ddots}}}}                \\
				     & \text{} &                                                                                                                                                  \\

				     & =       & 1+\cfrac{c_{1}z}{1-(c_{1}+c_{2})z-\cfrac{c_{2}c_{3}z^2}{1-(c_{3}+c_{4})z-\cfrac{c_{4}c_{5}z^2}{1-(c_{5}+c_{6})z-\cfrac{c_{6}c_{7}z^2}{\ddots}}}} \\
				     & \text{} &
			\end{array}
		\]
	\end{rm}
\end{proposition}

\begin{proposition}\label{cat-frac}
	On a
	\[C(z) = \cfrac{1}{1-z-\cfrac{z^2}{1-2z-\cfrac{z^2}{\ddots}}} \]
\end{proposition}
Preuve:
Comme $C(z)$ est solution de l'équation $zx^{2} - x +1 =0$, alors on a
\[
	C(z) = \cfrac{C(z)}{1}=\cfrac{C(z)}{C(z)-zC^2(z)}=\cfrac{1}{1-zC(z)}= \cfrac{1}{1-\cfrac{z}{1-zC(z)}} = \cfrac{1}{1-\cfrac{z}{1-\cfrac{z}{\ddots}}}
\]
On voit que $C(z)$ est une $S$-fraction avec $c_{i}=1$ ($\forall i \geq 1$).
D'où le résultat sera obtenu en utilisant la \Cref{j-frac}. \hspace{10pt} $\blacksquare$

% \section{Développement en fraction continue de $\Gamma_{n}$}
\section{Chemins de Motzkin valué}
\begin{definition}
	\begin{rm}
		Un 2-chemin de Motzkin valué est un couple (c,p) où $c = c_{1}c_{2}\cdots c_{n}$\\ et $p = p_{1}p_{2}\cdots p_{n}$
		vérifient les conditions suivantes:
		\begin{itemize}
			\item[1.] $c \in \Gamma_{n}$
			\item[2.] p est le poids associé à c
			\item[3.] $0\leq p_{i}\leq \gamma_{i-1}$, si $c_{i}=m\text{ ou }c_{i}=b$
			\item[4.]  $0\leq p_{i}\leq \gamma_{i-1} - 1$, si  $c_{i}=d\text{ ou }c_{i}=r$
		\end{itemize}
	\end{rm}
\end{definition}
Dans la suite, nous adopterons la notation du cours de combinatoire en S10 pour désigner l'ensemble des 2-chemins de Motzkin valués en $n$ pas, noté $HL(n)$. Cette notation sera employée ultérieurement.
Soit $c \in \Gamma_{n, 2}^{i}$ et $j \in [n]$. Si $c_{j} = m$ (resp d, b, r), alors on affecte le poids $m_{\gamma_{j-1}}$ (resp
$d_{\gamma_{j-1}}$, $b_{\gamma_{j-1}}$, $r_{\gamma_{j-1}}$) au $j^e$-pas où les  $m_{\gamma_{j-1}}$, $d_{\gamma_{j-1}}$,
$b_{\gamma_{j-1}}$ et  $r_{\gamma_{j-1}}$ sont des éléments d'un anneau commutatif.\vspace{10pt}\\
Dans toute la suite, on pose $H_{i,n} = \underset{c\in \Gamma_{n,2}^{i}}{\sum}w(c)$ où $w(c) = \underset{c_{j}=m}{\prod}
	m_{\gamma_{j-1}}\text{ } \underset{c_{j}=d}{\prod}d_{\gamma_{j-1}}\text{ }\underset{c_{j}=b}{\prod}
	b_{\gamma_{j-1}}\text{ }\underset{c_{j}=r}{\prod}r_{\gamma_{j-1}}$.\\
\begin{proposition} \label{weight-tab}
	On a:
	\[
		\begin{cases}
			H_{i,0} & =0 \text{ si }i\geq 1                                                                  \\
			H_{0,n} & =b_{0}H_{0,n-1}+d_{1}H_{1,n-1}                                                         \\
			H_{i,n} & =0 \text{ si } i>n                                                                     \\
			H_{i,n} & =m_{i-1}H_{i-1,n-1}+(b_{i}+r_{i})H_{i,n-1}+d_{i+1}H_{i+1,n-1} \text{ si }1\leq i\leq n \\
		\end{cases}
	\]
	avec la convention $H_{0,0}=1$
\end{proposition}

Preuve:
Soit $c \in \Gamma_{n,2}^{i}$. Il est clair que si $i\geq 1$ et $n=0$ alors $w(c)=0$ ou encore $H_{i, 0}=0$. \\
% Pour tout $i \geq 1$, l'ensemble $\Gamma_{0,2}^{i}=\emptyset$. On en déduit que $w(c)=0$. D'où $H_{i, 0} = 0$.
De plus, comme $\Gamma_{n, 2}^{0}=\Gamma_{n}$ alors, on a $c_{n}\neq r$ et $c_{n}\neq m$.\vspace{5pt}\\
D'où
\[
	\begin{array}{l l l}
		H_{0, n} & =       & \underset{\underset{c_{n}=b}{c\in \Gamma_{n,2}^{0}}}{\sum}w(c)
		\text{ }+\text{ }\underset{\underset{c_{n}=d}{c\in \Gamma_{n,2}^{0}}}{\sum}
		w(c)                                                                                        \\
		         & \text{} &                                                                        \\
		         & =       & \underset{c\in \Gamma_{n-1,2}^{0}}{\sum}w(c)b_{\gamma_{n-1}} \text{ }+
		\text{ }\underset{c\in \Gamma_{n-1,2}^{1}}{\sum}w(c)d_{\gamma_{n-1}}                        \\
		         & \text{} &                                                                        \\
		         & =       & \underset{c\in \Gamma_{n-1,2}^{0}}{\sum}w(c)b_{0} \text{ }+\text{ }
		\underset{c\in \Gamma_{n-1,2}^{1}}{\sum}w(c)d_{1}                                           \\
		         & \text{} &                                                                        \\
		         & =       & b_{0}H_{0,n-1}+d_{1}H_{1,n-1}
	\end{array}
\]\\Ensuite, on en déduit de la
définition de $\Gamma_{n,2}^{i}$ que si $i>n$ alors $\Gamma_{n,2}^{i}$ serait vide car on n'atteint jamais le coordonné $(n,i)$ en $n$ pas. Cela implique que pour tout $c \in \Gamma_{n,2}^{i}$, $H_{i,n}=0$ si $i>n$.\vspace{10pt}\\ Enfin, soit $c \in \Gamma_{n,2}^{i}$ tel que $1\leq i\leq n$.\\
On a:
$$
	\gamma_{n-1} = \begin{cases}
		i-1 & \text{ si }c_{n}=m                       \\
		i   & \text{ si }(c_{n}=b \text{ ou } c_{n}=r) \\
		i+1 & \text{ si }c_{n}=d
	\end{cases}
$$\\
On note $c^{(1)}$ le chemin obtenu à partir de $c$ en
supprimant sa dernière lettre. \\On déduit alors la relation suivante:
$$
	w(c) = \begin{cases}
		w(c^{(1)})m_{i-1} & \text{ si }c_{n}=m \\
		w(c^{(1)})b_{i}   & \text{ si }c_{n}=b \\
		w(c^{(1)})r_{i}   & \text{ si }c_{n}=r \\
		w(c^{(1)})d_{i+1} & \text{ si }c_{n}=d
	\end{cases}
$$\vspace{5pt}\\
Par conséquent,
\[
	\begin{array}{r c l}
		H_{i,n} & =       & \underset{c \in \Gamma_{n,2}^{i}}{\sum}w(c)                                              \\
		        & \text{} &                                                                                          \\
		        & =       & \underset{\scriptstyle\underset{c_{n}=m}{ c \in \Gamma_{n,2}^{i}}}{\sum}w(c)+
		\underset{\scriptstyle \underset{c_{n}=b}{c\in \Gamma_{n,2}^{i}}}{\sum}w(c)+
		\underset{\scriptstyle \underset{c_{n}=r}{c\in \Gamma_{n,2}^{i}}}{\sum}w(c)+
		\underset{\scriptstyle \underset{c_{n}=d}{c\in \Gamma_{n,2}^{i}}}{\sum}w(c)                                  \\
		        & \text{} &                                                                                          \\
		        & =       & m_{i-1}\underset{c^{(1)}\in \Gamma_{n-1,2}^{i-1}}{\sum}w(c^{(1)})+b_{i}\underset{c^{(1)}
			\in \Gamma_{n-1,2}^{i}}{\sum}w(c^{(1)})+r_{i}\underset{c^{(1)}\in \Gamma_{n-1,2}^{i}}{\sum}
		w(c^{(1)})+d_{i+1}\underset{c^{(1)} \in \Gamma_{n-1,2}^{i+1}}{\sum}w(c^{(1)})                                \\
		        & \text{} &                                                                                          \\
		        & =       & m_{i-1}H_{i-1,n-1}+(b_{i}+r_{i})H_{i,n-1}+d_{i+1}H_{i+1,n-1}\hspace{5pt}
	\end{array}
\]D'où le résultat $\blacksquare$\vspace{15pt}\\
Dans toute la suite, on pose $H_{i}(t) = \underset{n\geq 0}{\sum}H_{i, n}t^{n}$.
\begin{proposition}\label{H-frac} On a:
	\[
		\begin{array}{r c l}
			H_{0}(t)=\cfrac{1}{1-b_{0}t-\cfrac{m_{0}d_{1}t^2}{1-(b_{1}+r_{1})t-\cfrac{m_{1}d_{2}t^2}
			{1-(b_{2}+r_{2})t-\cfrac{m_{2}d_{3}t^2}{\ddots}}}}
		\end{array}
	\]
\end{proposition}
Preuve:
En utilisant la \Cref{weight-tab}, on a
$$
	\begin{array}{r c l}
		H_{0}(t) & =       & 1+\underset{n\geq1}{\sum}H_{0,n}t^n                                                 \\
		         & \text{}                                                                                       \\
		         & =       & 1+b_{0}\underset{n\geq1}{\sum}H_{0,n-1}t^n+d_{1}\underset{n\geq1}{\sum}H_{1,n-1}t^n \\
		         & \text{}                                                                                       \\
		         & =       & 1+b_{0}tH_{0}(t)+d_{1}tH_{1}(t)                                                     \\
		         & \text{}                                                                                       \\
		         & =       & \cfrac{1}{1-b_{0}t-d_{1}t\cfrac{H_{1}(t)}{H_{0}(t)}}
	\end{array}
$$
De plus, pour tout $i\geq1$,
\[
	\begin{array}{r c l}
		H_{i}(t) & =       & \underset{n\geq0}{\sum}H_{i,n}t^n                                                                     \\
		         & \text{} &                                                                                                       \\
		         & =       & \underset{n\geq1}{\sum}H_{i,n}t^n                                                                     \\
		         & \text{} &                                                                                                       \\
		         & =       & \underset{n\geq1}{\sum}m_{i-1}H_{i-1,n-1}t^n+(b_{i}+r_{i})\underset{n\geq1}{\sum}H_{i,n-1}t^n+d_{i+1}
		\underset{n\geq}{\sum}H_{i+1,n-1}t^n                                                                                       \\
		         & \text{} &                                                                                                       \\
		         & =       & m_{i-1}tH_{i-1}(t)+(b_{i}+r_{i})tH_{i}(t)+d_{i+1}H_{i+1}(t)
	\end{array}
\]
Cela implique que :\\
\[
	\cfrac{H_{i}(t)}{H_{i-1}(t)}=\cfrac{m_{i-1}t}{1-(b_{i}+r_{i})t-d_{i+1}t\cfrac{H_{i+1}(t)}{H_{i}(t)}}
\]
Par conséquent:
\[
	\begin{array}{r c l}
		H_{0}(t) & =       & \cfrac{1}{1-b_{0}t-d_{1}t\cfrac{m_{0}t}{1-(b_{1}+r_{1})t-d_{2}t\cfrac{H_{2}(t)}{H_{1}(t)}}}                   \\
		         & \text{} &                                                                                                               \\
		         & =       & \cfrac{1}{1-b_{0}t-\cfrac{m_{0}d_{1}t^2}{1-(b_{1}+r_{1})t-\cfrac{m_{1}d_{2}t^2}{1-(b_{2}+r_{2})t-\cfrac{m_{2}
			d_{3}t^2}{\ddots}}}}
	\end{array}
\]
D'où le résultat. \hspace{5pt}$\blacksquare$

\begin{corollaire} \label{H0(t)}
	On a $H_{0}(t) = 1 + \underset{n\geq 1}{\sum}t^{n}\text{ }\underset{c\in \Gamma_{n}}{\sum} w(c)$
\end{corollaire}
Preuve: Le résultat se déduit du fait que  $H_{0,n} = \underset{c\in \Gamma_{n,2}^{0}}{\sum}w(c) = \underset{c\in \Gamma_{n}}{\sum}w(c)$

\begin{proposition}\label{gamma-frac}
	On a \[1+\underset{n \geq 1}{\sum}|\Gamma_{n}|t^n = \cfrac{1}{1-t-\cfrac{t^2}{1-2t-\cfrac{t^2}{\ddots}}}\]
\end{proposition}
Preuve:
Comme $|\Gamma_{n}| = \underset{c\in \Gamma_{n}}{\sum}1 $, alors on en déduit
$m_{\gamma_{i-1}}=d_{\gamma_{i-1}}=r_{\gamma_{i-1}}=b_{\gamma_{i-1}}=1$. Ainsi, en
utilisant la \Cref{H-frac} et le \Cref{H0(t)}, on obtient:\\
$
	1+\underset{n \geq 1}{\sum}|\Gamma_{n}|t^n = 1 + \underset{n \geq 1}{\sum}t^n \left(
	\underset{c\in \Gamma_{n}}{\sum}1  \right) = \cfrac{1}{1-t-\cfrac{t^2}{1-2t-
			\cfrac{t^2}{\ddots}}}
$

\begin{corollaire}
	On a $|\Gamma_{n}|=C_{n}$.
\end{corollaire}
Preuve: En utilisant la \Cref{cat-frac}. et \Cref{gamma-frac}. on
obtient le résultat.\\

\section{Développement en fraction continue de $F(t)$}
On note par $F(t)$ la fonction génératrice ordinaire des nombres de Fine.
i.e $F(t) = \underset{n \geq 0}{\sum}F_{n}t^{n}$ avec la convention $F_{0}=1$. Le développement en fraction continue de $F(t)$ nous permet de trouver quelques résultats sur les relations entre les nombres de Catalan $C_{n}$ et les nombres de Fine $F_{n}$ dont lesquels nous allons faire dans cette section.\vspace{10pt}
\begin{proposition} \label{Fn-frac}
	On a $$F(t) = \cfrac{1}{1 - \cfrac{t^2}{1 - 2t - \cfrac{t^2}{\ddots}}}$$\vspace{5pt}
\end{proposition}
Preuve:
D'abord, le nombre de 2-chemins de Motzkin de longueur n est égal à $\underset{c \in \Gamma_{n}}{\sum}1 = \sum\limits_{c\in \Gamma_{n}}w(c)$. Cela implique que chaque pas de $c$ est associé à un poids égal à 1.\\
Dans le cas de Fine, on affecte à chaque palier bleu le poids 0 (resp. 1) si son niveau est nul (resp. non
nul).
Ainsi
\[
	\begin{array} {l l l}
		F_{n} & =       & \underset{c \in \Gamma_{n}}{\sum}w(c)                                                                    \\
		      & \text{} &                                                                                                          \\
		      & =       & \sum\limits_{c\in \Gamma_{n}}\underset{c_{i}=m}{\prod}m_{\gamma_{i-1}}\text{ } \underset{c_{i}=d}{\prod}
		d_{\gamma_{i-1}}\text{ }\underset{c_{i}=b}{\prod}b_{\gamma_{i-1}}\text{ }\underset{c_{i}=r}
		{\prod}r_{\gamma_{i-1}}                                                                                                    \\
	\end{array}
\]
\text{}\vspace{10pt}\\
où $m_{j}=1, d_{j}=1, r_{j}=1$ et $b_{j} = \begin{cases}
		1 & \text{ si } j\neq 0 \\
		0 & \text{ si } j=0
	\end{cases}$\\
En utilisant la \Cref{H-frac}, on obtient le résultat. \hspace{10pt}$\blacksquare$
\begin{proposition} \label{CatFinGenRelation}
	On a \[F(t)=\cfrac{1}{2+t}(1+C(t))\]
\end{proposition}
{Preuve}:
D'abord, on pose \[\Delta(t)=\cfrac{t^2}{1-2t-\cfrac{t^2}{1-2t-\cfrac{t^2}{\ddots}}}\]
On a $C(t)=\cfrac{1}{1-t-\Delta(t)}$ et $F(t)=\cfrac{1}{1-\Delta(t)} $ ou encore $F(t)=\cfrac{C(t)}{1+tC(t)} $. \vspace{5pt}\\
Ainsi, on obtient:
\[
	\begin{array} {c c l}
		F(t) & = & \cfrac{\cfrac{1-\sqrt{1-4t}}{2t}}{1+\cfrac{1-\sqrt{1-4t}}{2t}} =\cfrac{1-\sqrt{1-4t}}{t(3-\sqrt{1-4t})}=\cfrac{2-2\sqrt{1-4t}+4t}{t(8+4t)}=\cfrac{1+\cfrac{1-\sqrt{1-4t}}{2t}}{2+t}
	\end{array}
\]
$\blacksquare$
\begin{proposition}
	Pour tout $n\geq 2$,\[ F_{n}=\cfrac{1}{2}\sum_{p=0}^{n-2}(-\frac{1}{2})^p C_{n-p}\]
\end{proposition}
Preuve:
Il suffit d'extraire le coefficient de $t^{n}$ dans la relation de la
\Cref{CatFinGenRelation}.\\
On a
\[
	\begin{array}{c c l}
		F(t) & =       & \cfrac{1}{2+t}(1+C(t))=\cfrac{1}{2}\cfrac{1}{1+\cfrac{t}{2}}\left(1+C(t)\right)
		=\cfrac{1}{2}\left(\sum_{p\geq 0}(-\cfrac{1}{2})^p t^p\right)\left(1+\sum_{m\geq 0}C_{m}t^m\right)                        \\
		     & \text{} &                                                                                                          \\
		     & =       & \cfrac{1}{2} \left[ \sum_{p\geq 0}(-\cfrac{1}{2})^p t^p + \underset{k,m\geq0}{\sum}C_{m}(-\frac{1}{2})^k
		t^{m+k}\right]=\cfrac{1}{2}\left[\sum_{p\geq 0}(-\cfrac{1}{2})^p t^p +\underset{n\geq0}{\sum}
			t^n\underset{k=0}{\overset{n}{\sum}}(-\cfrac{1}{2})^k C_{n-k}\right]
	\end{array}
\]
\text{}\vspace{5pt}\\
On en déduit:
\[
	\begin{array}{c c l}
		F_{n} & =       & \cfrac{1}{2}\left[(-\cfrac{1}{2})^n + \underset{k=0}{\overset{n}{\sum}}(-\cfrac{1}{2})^k C_{n-k}
			\right]=\cfrac{1}{2}\left[(-\cfrac{1}{2})^n + (-\cfrac{1}{2})^n +(-\cfrac{1}{2})^{n-1} +\underset{k=0}
		{\overset{n-2}{\sum}}(-\cfrac{1}{2})^k C_{n-k}\right]                                                                \\
		      & \text{} &                                                                                                    \\
		      & =       & \cfrac{1}{2}\left[2(-\frac{1}{2})^n + (-\cfrac{1}{2})^{n-1} +\underset{k=0}{\overset{n-2}{\sum}}(-
			\cfrac{1}{2})^k C_{n-k}\right]=\cfrac{1}{2}\underset{k=0}{\overset{n-2}{\sum}}(-\cfrac{1}{2})^k C_{n-k}
		\hspace{10pt}\blacksquare
	\end{array}
\]
\text{}\vspace{5pt}
\begin{proposition}\label{cnfn}
	On a \[\forall n\geq 1, C_{n}=2F_{n}+F_{n-1}\]
\end{proposition}
\text{}\\
Preuve:
On en déduit de la \Cref{CatFinGenRelation} que $C(t)=(2+t)F(t)-1$. En remplaçant $F(t)$ par sa fonction génératrice ordinaire, on obtient
\[C(t)= (2+t)\underset{n\geq0}{\sum}F_{n}t^n -1=\sum_{n\geq 0}2F_{n}t^n+\sum_{n\geq 0} F_{n}
	t^{n+1}-1=\sum_{n\geq 1}2F_{n}t^n+\sum_{n\geq 1} F_{n-1}t^{n}+1\]
D'où le résultat \hspace{5pt}$\blacksquare$ \vspace{10pt}\\
Rappelons qu'un chemin de Dyck de longueur 2n est un chemin dans $\mathbb{N}\times \mathbb{N}$ de $(0, 0)$ vers $(2n, 0)$ formé par les pas montées $(1,1)$
et descentes $(1,-1)$. \\
Il est bien connu que le nombre de Catalan $C_{n}$ est égal au nombre de chemins de Dyck
de longueur 2n. Dans toute la suite, on pose $\overline{\text{Dyck}}(n)$ l'ensemble des chemins de Dyck qui
vérifient la condition\\ si $p_{i}=m \text{ et } \gamma_{i-1}=0$, alors  $p_{i+1}=m$.

\begin{proposition} \label{bij-DyckBar}
	La transformation $\theta: \mathcal{F}_{n} \longrightarrow  \overline{\text{Dyck}}(n)$; $c \longrightarrow p$ définit comme suit:\\pour tout $i \in [n]$,
	$$
		p_{2i-1}p_{2i}=\begin{cases}
			mm & \text{ si } c_{i}=m \\
			dd & \text{ si } c_{i}=d \\
			md & \text{ si } c_{i}=b \\
			dm & \text{ si } c_{i}=r \\
		\end{cases}
	$$
	est une application bijective.
\end{proposition}
Preuve: Soit $c\in \mathcal{F}_{n}$ et $\theta(c):=p$.
Nous allons voir que $p\in \overline{\text{Dyck}}(n) $.\\
Par définition de $\mathcal{F}_{n}$, on a $|c|_{d}=|c|_{m}$ ou encore $\#\{i; p_{2i-1}p_{2i}=mm\}=\#\{i; p_{2i-1}p_{2i}=dd\}$. Par construction, à chaque palier
bleu (resp. palier rouge) de $c$ on a associé à un sous-chemin $md$ (resp $dm$) dans $p$; ce qui implique que $|p|_{m}=|p|_{d}$. Nous allons voir que pour tout $j\leq 2n$, $|p_{1} \cdots p_{j}|_{m}\geq |p_{1} \cdots p_{j}|_{d}$.\\
Fixons $j$ et distinguant le cas où $j$ impair et $j$ pair. Supposons que $j$ est pair. Il existe $i\in [n]$ tel que $j=2i$. On a \vspace{5pt}\\
\[
	|p_{1} \cdots p_{2i-1}p_{2i}|_{m} = |c_{1} \cdots c_{i}|_{m}\geq |c_{1} \cdots c_{i}|_{d} = |p_{1} \cdots p_{2i-1}p_{2i}|_{m}
\]
\text{}\vspace{5pt}\\
Supposons ensuite que $j$ est impair. Il existe $i\in [n]$ tel que $j = 2i-1$.\\ Nous allons distinguer le cas où $c_{i} = m$ (resp. $c_{i}=r, c_{i}=b, c_{i}=d$).\\
Si $c_{i}=m$, alors on a \vspace{5pt}\\
\[
	\begin{array} {l l l}
		| p_{1}\cdots p_{2i-1} |_{m} & =    & | p_{1}\cdots p_{2i-1}p_{2i} |_{m} -1 \\
		                             & =    & | c_{1}\cdots c_{i} |_{m} -1          \\
		                             & =    & | c_{1}\cdots c_{i-1} |_{m}           \\
		                             & \geq & | c_{1}\cdots c_{i-1} |_{d}           \\
		                             & =    & | p_{1}\cdots p_{2i-2} |_{d}          \\
		                             & =    & | p_{1}\cdots p_{2i-2}p_{2i-1} |_{d}
	\end{array}
\]
\text{} \vspace{5pt}\\
Si $c_{i}=r$, alors on a \text{}\vspace{5pt}\\
\[
	\begin{array} {l l l}
		| p_{1}\cdots p_{2i-1} |_{m} & =    & | p_{1}\cdots p_{2i-2}|_{m}        \\
		                             & =    & | c_{1}\cdots c_{i-1} |_{m}        \\
		%  & =    & | c_{1}\cdots c_{i-1} |_{m}           \\
		                             & \geq & | c_{1}\cdots c_{i-1} |_{d}        \\
		                             & =    & | c_{1}\cdots c_{i} |_{d}          \\
		                             & =    & | p_{1}\cdots p_{2i-1}p_{2i} |_{d} \\
		                             & =    & | p_{1}\cdots p_{2i-1} |_{d}
	\end{array}
\]
\text{} \vspace{5pt}\\
Si $c_{i}=d$ ou $c_{i}=b$, alors on a \text{}\vspace{5pt}\\
\[
	\begin{array} {l l l}
		| p_{1}\cdots p_{2i-1} |_{m} & =    & | p_{1}\cdots p_{2i}|_{m}          \\
		                             & =    & | c_{1}\cdots c_{i} |_{m}          \\
		%  & =    & | c_{1}\cdots c_{i-1} |_{m}           \\
		                             & \geq & | c_{1}\cdots c_{i} |_{d}          \\
		%  & =    & | c_{1}\cdots c_{i} |_{d} \\
		                             & =    & | p_{1}\cdots p_{2i-1}p_{2i} |_{d} \\
		                             & >    & | p_{1}\cdots p_{2i-1} |_{d}
	\end{array}
\]
Par conséquent, pour tout $j\leq 2n$, $|p_{1} \cdots p_{j}|_{m}\geq |p_{1} \cdots p_{j}|_{d}$. Ainsi, $p\in \text{Dyck}(n)$
\vspace{20pt}\\
D'autre part, soit $j$ tel que $p_{j}=m \text{ et } \gamma_{j-1}=0$. Nécéssairement, $j$ est impair i.e il existe $i\leq n$ tel que $j=2i-1$. Cela implique que $|p_{1} \cdots p_{2i-2}|_{m} = |p_{1} \cdots p_{2i-2}|_{d}$ ou encore $|
	c_{1} \cdots c_{i-1}|_{m} = |c_{1} \cdots c_{i-1}|_{d}$. Comme $c$ est un élément $
	\mathcal{F}_{n}$, alors $c_{i}$ ne peut pas être égal à $d$, $r$ ou $b$. D'où, $c_{i}=m$
ou encore, $p_{2i}=p_{j+1}=m$. Ainsi $p\in \overline{\text{Dyck}}(n) $. Ce qui prouve que $\theta$ est bien définie.\vspace{15pt}\\
Montrons qu'elle est bijective. Soit $p\in \overline{\text{Dyck}}(n) $ et c son antécédent par $\theta$	(s'il existe).\\ On note que $\#\{i; p_{2i-1}p_{2i}=mm\} = \#\{i; p_{2i-1}p_{2i}=dd\}$. Cela implique que $|c|_{m} = |c|_{d}$.\\
Fixons $j\in [n]$ et montrons que $| c_{1}\cdots c_{j} |_{m}\geq | c_{1}\cdots c_{j} |_{d}$. On a \vspace{5pt}
\[
	| c_{1}\cdots c_{j} |_{m} = |  p_{1}\cdots p_{2j-1}p_{2j} |_{m} \geq | p_{1}\cdots p_{2j-1}p_{2j} |_{d} = | c_{1}\cdots c_{j} |_{d}
\]
\text{}\vspace{5pt}
% Soit $j \in [n]$ tel que $|p_{1}\cdots p_{2j-1}p_{2j}|_{m} \geq p_{1}\cdots p_{2j-1}p_{2j}|_{d}$. \\
% On en déduit  $\#\{i; i\leq j \text{ tel que  } p_{2i-1}p_{2i}=mm\} \geq \#\{i; i \leq j \text{ tel que } p_{2i-1}p_{2i}=dd\}$ ou encore $|c_{1}\cdots c_{j}|_{m} \geq |c_{1}\cdots c_{j}|_{d} $
Enfin montrons que $c$ ne contient pas ni palier bleu ni palier rouge de niveau zéro. \\
Soit $i\in [n]$. Si $c_{i}=b$ (resp $c_{i}=r$) alors on a $\gamma_{i-1}\neq 0$ parce que $p_{2i-1}p_{2i}=md$ (resp. $p_{2i-1}p_{2i}=dm$) avec le niveau du pas $p_{2i}$ différent de $1$ (resp. différent de 0).\\
Dès lors, on a $c\in \mathcal{F}_{n}$. \hspace{10pt}$\blacksquare$
\begin{proposition}
	On a: $ F_{n} = \underset{i=2}{\overset{n}{\sum}}C_{i-1}F_{n-i} $
\end{proposition}
$\underline{\textit{Preuve}}$:\\
Soit $i\in [n]$. On pose $\overline{\text{Dyck}}_{i}(n)$ l'ensemble des chemins $p$ élément de $\overline{\text{Dyck}}(n)$ qui vérifient les conditions suivantes:
$|p_{1}p_{2}\cdots p_{2i}|_{m} = |p_{1}p_{2}\cdots p_{2i}|_{d} \text{ et } \forall j<2i,
	|p_{1}p_{2}\cdots p_{j}|_{m}>|p_{1}p_{2}\cdots p_{j}|_{d}$. Nécéssairement $i\neq 1$. Soit $p \in \overline{\text{Dyck}}_{i}(n)$.
On a $p_{1} = p_{2} = m$,  $p_{2i-1} = p_{2i} = d$, $p_{2}\cdots p_{2i-1} \in \text{Dyck}(i-1) $ et \\ $p_{2i+1} \cdots p_{2n} \in \overline{\text{Dyck}}(n-i)$. Donc il existe une bijection de $\overline{\text{Dyck}}_{i}(n)$ sur $ \text{Dyck}(i-1) \times
	\overline{\text{Dyck}}(n-i)$.\\
De plus $\underset{i=2}{\overset{n}{\bigcup}}\overline{\text{Dyck}}_{i}(n) = \overline{\text{Dyck}}
	(n)$. En
utilisant la \Cref{bij-DyckBar}, on a le résultat. \hspace{10pt} $\blacksquare$.
%\section{Chemins de Dyck}
























