\documentclass{report}


\begin{document}
    \section*{Prélude}
    \textit{Bonjour à tous,}

    \vspace{7pt}\textit{C'est un mélange d'enthousiasme et de gratitude que je me présente devant vous aujourd'hui pour présenter mon mémoire intitulé \\\textbf{Quelques interprétations des nombres de Fine}. Ce moment marque non seulement la culmination de nombreuses heures de travail et de reflexion mais également une étape importante dans mon parcours académique.}

    \vspace{7pt}\textit{Je tiens à exprimer ma profonde gratitude envers le Pr. \textbf{Arthur RANDRIANARIVONY} pour son soutien constant et ses conseils éclairés. Mes sincères remerciements s'adressent également au Jury \textbf{Mr. Joseph RAKOTONDRALAMBO} et à l'examinateur \textbf{Mr. Hanjarivo LALAHARISON} pour leur temps. Je souhaite également exprimer ma reconnaissance envers mes collègues, mes amis et ma famille pour leur encouragement indéfectible.}

    \vspace{7pt}\textit{Sans plus tarder, je vous invite à vous joindre à moi dan s cette exploration intellectuelle, dans l'espoir que cette présentation puisse susciter des réflexions nouvelles et enrichir notre compréhension des nombres de Fine}.\\
    \textit{Merci}\vspace{10pt}\\
    \textit{Pour commencer, Cette présentation se divise en deux grandes parties. La prémiere se focalise sur quelques définitions et résultats que nous avons déjà vus dans le semestre 9 du parcours Combinatoire et Optimissation}.\\
    \textit{Dans la deuxième partie, on va explorer un peu plus profonde les nombres de Fine.}
    \section*{Définitions et quelques résultats préliminaire}
    Premièrement, nous avons vu les fractions continues. Et deux formes de fractions continues nous intêressent dans ce mémoire et qui sont: $\cdots$\\ 
    Ces deux formes de fractions continues nous servent de base à la démonstration du développement en fraction continue des nombres de Fine.\vspace{7pt}\\
    Ensuite, les nombres de Catalan. Nous allons les utilisés plus tard pour établir une relation de récurrence avec les nombres de Fine. On voit le développement en fraction continue de la fonction génératrice ordinaire de nombres de Catalan est de la forme J-fraction.\vspace{7pt}\\
    Enfin, les chemins de Motzkin. Trois types de chemins a été explorés dans ce mémoire et ils jouent tous un rôle crucial dans l'interprétation des nombres de Fine et qui sont: $\cdots$\\
    Le résultats suivant nous montre que le développement en fraction continue des nombres de 2-chemins de Motzkin est de forme J-fraction. Et ce dernier est égale au développement en fraction continue dees nombres de Catalan. D'où le corollaire suivant $\cdots$\vspace{7pt}\\
    Maintenant que nous avons introduit les concepts clés et les fondements de notre étude, un premier aperçu sur les nombres de Fine sera essentiel. Les résultats suivants seront utilisés dans la prochaine partie afin qu'on puisse montrer qu'on a bien les nombres que Terrence Fine a mentionné dans son article sur l'extrapolation d'une fonction.\\
    Par définition, $\cdots$\\
    Et on a la proposition suivant qui nous montre la relation entre $F(z)$ et $C(z)$.\\
    L'extraction du coéfficient de $z^{n}$ dans $F(z)$ et $C(z)$ nous amène aux deux relations de récurrence suivantes: $\cdots$\\
    De plus, on a une corréspondance bijective entre l'ensemble des chemins de Fine et les chemins de Dyck qui vérifient la condition ci-dessous.
    Cette bijection nous sert comme un pont entre le nombre que Terrence a défini dans son article et le nombre $F_{n}$ qui sera utilisé plus tard comme le nombre de Fine.\vspace{7pt}\\
    Maintenant nous allons commencer la dernière partie de cette présentation.
    \section*{Interprétation des nombres de Fine}
    Terrence Fine a introduit le concept de relation de similarité et qui est par définition $\cdots$\\
    Par définition, le nombre de Fine correspond à la cardinalité de l'ensemble des relations de similarité sans point isolé. Notre premier objectif est de démontrer que le nombre $F_{n}$ dans la première partie correspond à ce nombre défini par Terrence Fine. A cette fin, deux bijections seront nécessaire et qui sont $\cdots$\vspace{7pt}\\
    Sans oublier les permutations et leurs aspects, nous allons voir quelques résultats entre ces dernières et les nombres de Fine. On a $\cdots$\vspace{7pt}\\
    Enfin, la dernière ligne de cette présentation sera consacrée à la demarche pour obtenir les deux propositions suivantes dans la première nous montre que $\cdots$\\
    Une nouvelle concept nous sera utile pour montrer la première proposition et qui est les mots de Catalan. Ces derniers n'est autre que la transformation des chemins de Dyck. Plusieurs résultats nous seront utiles pour trouver notre fameux relation entre les nombres de Fine et les premutations évitant un motif, dont lequel chaque résultat se déduit du précédent dans une séquence de déduction en cascade. Ce qui nous intêresse, c'est le coéfficient $b(n, k)$ car on a pu trouver une relation de récurrence entre ce coéfficient et le nombre $c_{n, k}$ dans le triangle de Catalan défini comme suit: $\cdots$

\end{document}